\documentclass[10pt]{beamer}

% \usepackage{define}
\usepackage{animate}

\usetheme{CCFD}
\usepackage{color}
\definecolor{gray97}{gray}{.90}
\definecolor{gray75}{gray}{.75}
\usepackage{listings}
\lstset{frame=Ltb,
     framerule=0pt,
     aboveskip=0cm,
     framextopmargin=0pt,
     framexbottommargin=0pt,
     framexleftmargin=0cm,
     framesep=0pt,
     rulesep=0pt,
     backgroundcolor=\color{gray97},
     rulesepcolor=\color{black},
     language=C,
           basicstyle=\ttfamily\scriptsize,
           keywordstyle=\color{blue}\ttfamily,
           stringstyle=\color{red}\ttfamily,
           commentstyle=\color{green}\ttfamily,
          breaklines=true,
          }
\lstdefinestyle{consol}
   {basicstyle=\scriptsize\bf\ttfamily,
    backgroundcolor=\color{gray75},
}
\resetcounteronoverlays{lstnumber}

\newcommand{\tabitem}{%
  \usebeamertemplate{itemize item}\hspace*{\labelsep}}

\usepackage{tikz}
\usetikzlibrary{calc,shapes,arrows.meta}

\eventtitle{Computer Science I}
\title{Lecture 12\\Structures}
\date{}

\setbeamertemplate{blocks}[rounded][shadow=true]
\setbeamertemplate{navigation symbols}[]

\begin{document}

\frame{
    \titlepage
}

\section{Introduction}

\begin{frame}[fragile]
  \frametitle{Simple data types}
  \framesubtitle{}  
\centering

\begin{itemize}
  \item Up to now we used simple, built-in data types, such as int, double, etc.
  \item Those could be grouped using static or dynamic arrays.
  \item Example: Write a program simulating motion of a couple of "robots"
  \begin{itemize}
    \item Each robot has a name, position, instant velocity and acceleration
    \item Each robot moves, according to its velocity
    \item Our program will get rather messy for larger number of robots
  \end{itemize}
  \item If our program becomes larger it will become difficult to handle
  \item Need a method to organize it better
  \item We know how to separate functionality,
  by dividing what program does in to different functions
  \item Structures are used to group data in to larger constructs,
  allowing a better overview of it
\end{itemize}

\end{frame}

\section{Structure}

\begin{frame}[fragile]
  \frametitle{Structure}
  \framesubtitle{User defined data type}  

\textbf{Structure} is a user defined data type available in C that allows the programmer to combine data items of different kinds.

\vspace{0.2cm}
Syntax:\vspace{0.1cm}
\begin{lstlisting}
struct structure_name {
   member_type member_mane;
   member_type member_mane;
   ...
   member_type member_mane;
} one or more structure variables;  
\end{lstlisting}

E.g.:\vspace{0.2cm}
\begin{lstlisting}
struct Robot {
   char  name[50]; //String identifying the robot, its name
   double x,y,vx,vy,ax,ay; //position, velocity, acceleration
   ...
} r1, r2; 
\end{lstlisting}
\end{frame}

\begin{frame}[fragile]
  \frametitle{Structure}
  \framesubtitle{User defined data type}  

Access to \textit{member} elements through '.', and $->$

\vspace{0.2cm}
Usage:\vspace{0.1cm}
\begin{lstlisting}
struct Robot {
   char  name[50];
   double x,y,vx,vy,ax,ay;
   ...
} r1, r2;

int main(){
  //r1 and r2 are allready defined and global
  r1.x = 8.0;
  
  struct Robot r3;
  
  struct Robot *p = &r3;
  p->y=5.0;
}

\end{lstlisting}

Write an example, define objects of type Robot, and acces data through '.' and '$->$'

\end{frame}

\begin{frame}[fragile]
  \frametitle{Structure}
  \framesubtitle{Use with functions}  

  Structures can be passed to functions, by value (copy) and with a pointer.
  \vspace{1cm}
  
  C language does not allow structures to 'posses' functions (in C++ this is allowed),
  but we still can add a pointer to a function ...
  \vspace{1cm}
  
  Write an example of a list data colloection using structures. What is a list?
\end{frame}


















\end{document}
